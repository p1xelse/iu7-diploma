\section*{ПРИЛОЖЕНИЕ А}
\section*{Модуль модели GoogleNet}
\addcontentsline{toc}{section}{ПРИЛОЖЕНИЕ А}
\setcounter{listing}{0}
\renewcommand{\thelisting}{A.\arabic{listing}} % Формат нумерации листингов (секционный)

\begin{code}
	\captionsetup{aboveskip=0pt, skip=-5mm}
	\captionof{listing}{Модель GoogleNet. Часть 1}
	\label{lst:googlenet}
	\inputminted
	[
	frame=single,
	framerule=0.5pt,
	framesep=20pt,
	fontsize=\footnotesize,
	tabsize=4,
	linenos,
	breaklines,
	numbersep=5pt,
	xleftmargin=10pt,
	firstline=1,
	lastline=21,
	]
	{text}
	{../src/train/model.py}
\end{code}
\clearpage

\begin{code}
	\captionsetup{aboveskip=0pt, skip=-5mm}
	\captionof{listing}{Модель GoogleNet. Часть 2}
	\label{lst:googlenet_2}
	\inputminted
	[
	frame=single,
	framerule=0.5pt,
	framesep=20pt,
	fontsize=\footnotesize,
	tabsize=4,
	linenos,
	breaklines,
	numbersep=5pt,
	xleftmargin=10pt,
	firstline=22,
	lastline=40,
	]
	{text}
	{../src/train/model.py}
\end{code}

\clearpage

\begin{code}
	\captionsetup{aboveskip=0pt, skip=-5mm}
	\captionof{listing}{Модель GoogleNet. Часть 3}
	\label{lst:googlenet_3}
	\inputminted
	[
	frame=single,
	framerule=0.5pt,
	framesep=20pt,
	fontsize=\footnotesize,
	tabsize=4,
	linenos,
	breaklines,
	numbersep=5pt,
	xleftmargin=10pt,
	firstline=41,
	lastline=74,
	]
	{text}
	{../src/train/model.py}
\end{code}

\clearpage

\begin{code}
	\captionsetup{aboveskip=0pt, skip=-5mm}
	\captionof{listing}{Модель GoogleNet. Часть 4}
	\label{lst:googlenet_4}
	\inputminted
	[
	frame=single,
	framerule=0.5pt,
	framesep=20pt,
	fontsize=\footnotesize,
	tabsize=4,
	linenos,
	breaklines,
	numbersep=5pt,
	xleftmargin=10pt,
	firstline=75,
	lastline=109,
	]
	{text}
	{../src/train/model.py}
\end{code}

\clearpage

\begin{code}
	\captionsetup{aboveskip=0pt, skip=-5mm}
	\captionof{listing}{Модель GoogleNet. Часть 5}
	\label{lst:googlenet_5}
	\inputminted
	[
	frame=single,
	framerule=0.5pt,
	framesep=20pt,
	fontsize=\footnotesize,
	tabsize=4,
	linenos,
	breaklines,
	numbersep=5pt,
	xleftmargin=10pt,
	firstline=110,
	lastline=142,
	]
	{text}
	{../src/train/model.py}
\end{code}

\clearpage

\begin{code}
	\captionsetup{aboveskip=0pt, skip=-5mm}
	\captionof{listing}{Модель GoogleNet. Часть 6}
	\label{lst:googlenet_6}
	\inputminted
	[
	frame=single,
	framerule=0.5pt,
	framesep=20pt,
	fontsize=\footnotesize,
	tabsize=4,
	linenos,
	breaklines,
	numbersep=5pt,
	xleftmargin=10pt,
	firstline=143,
	lastline=170,
	]
	{text}
	{../src/train/model.py}
\end{code}

\clearpage


\begin{code}
	\captionsetup{aboveskip=0pt, skip=-5mm}
	\captionof{listing}{Обучение модели. Часть 1}
	\label{lst:train_1}
	\inputminted
	[
	frame=single,
	framerule=0.5pt,
	framesep=20pt,
	fontsize=\footnotesize,
	tabsize=4,
	linenos,
	breaklines,
	numbersep=5pt,
	xleftmargin=10pt,
	firstline=1,
	lastline=35
	]
	{text}
	{../src/train/train.py}
\end{code}
\clearpage

\begin{code}
	\captionsetup{aboveskip=0pt, skip=-5mm}
	\captionof{listing}{Обучение модели. Часть 2}
	\label{lst:train_2}
	\inputminted
	[
	frame=single,
	framerule=0.5pt,
	framesep=20pt,
	fontsize=\footnotesize,
	tabsize=4,
	linenos,
	breaklines,
	numbersep=5pt,
	xleftmargin=10pt,
	firstline=36,
	lastline=63,
	]
	{text}
	{../src/train/train.py}
\end{code}
\clearpage

\begin{code}
	\captionsetup{aboveskip=0pt, skip=-5mm}
	\captionof{listing}{Обучение модели. Часть 3}
	\label{lst:train_3}
	\inputminted
	[
	frame=single,
	framerule=0.5pt,
	framesep=20pt,
	fontsize=\footnotesize,
	tabsize=4,
	linenos,
	breaklines,
	numbersep=5pt,
	xleftmargin=10pt,
	firstline=64,
	lastline=95,
	]
	{text}
	{../src/train/train.py}
\end{code}
\clearpage

\begin{code}
	\captionsetup{aboveskip=0pt, skip=-5mm}
	\captionof{listing}{Обучение модели. Часть 4}
	\label{lst:train_4}
	\inputminted
	[
	frame=single,
	framerule=0.5pt,
	framesep=20pt,
	fontsize=\footnotesize,
	tabsize=4,
	linenos,
	breaklines,
	numbersep=5pt,
	xleftmargin=10pt,
	firstline=96,
	lastline=140,
	]
	{text}
	{../src/train/train.py}
\end{code}
\clearpage

\pagebreak
\section*{ВВЕДЕНИЕ}
\addcontentsline{toc}{section}{ВВЕДЕНИЕ}

В последние годы такси стало полноценной частью общественного транспорта, особенно в мегаполисах.
В 2021 году количество занятых на постоянной основе в отрасли таксоперевозок составило 575 тысяч человек. Опросы показывают, что для пассажиров в вопросе выбора такси приоритетна не только цена поездки, но и безопасность \cite{wciom}.

Согласно исследованиям\cite{dtp} за 2017 год произошло 142000 ДТП, из которых 2917 с участием такси, при этом рост ДТП с участием такси составил порядка 17.5\%.

Исходя из приведенных выше фактов, можно сделать вывод, что изучение поведения водителей и прогнозирование потенциального риска автомобильных аварий являются критическими аспектами обеспечения безопасности дорожного движения. В связи с этим возникает необходимость разработки эффективных методов оценки безопасности водителя.

Для оценки безопасности водителей можно воспользоваться компьютерными технологиями, такие технологии могут быть полезны как для сервисов такси, так и для сервисов каршеринга.

Цель работы -- разработка метода оценки безопасности водителя с использованием глубоких нейронных сетей.

Для достижения поставленной цели требуется решить следующие задачи:
\begin{itemize}[leftmargin=1.6\parindent]
	\item[--] описать термины предметной области и формализовать задачу оценки безопасности водителя;
	\item[--] рассмотреть и сравнить методы классификации изображений;
	\item[--] спроектировать метод оценки безопасности водителя на основе глубоких нейронных сетей;
	\item[--] разработать программное обеспечение, реализующее данный метод;
	\item[--] провести исследование зависимости точности модели от размера входных данных и времени работы метода от объема входных данных.
\end{itemize}

\pagebreak
\section*{ВВЕДЕНИЕ}
\addcontentsline{toc}{section}{ВВЕДЕНИЕ}

В современном мире автомобили стали неотъемлемой частью повседневной жизни людей, и безопасность на дорогах становится все более актуальной проблемой.

Согласно статистике\cite{auto-acciden-stat} ежегодно около 3000 человек погибают в автокатастрофах из-за невнимательного вождения.

С помощью компьютерных технологий можно распознавать действия водителя по фотографии. Такие технологии могут быть полезны как для уведомления водителя, так и для выявления причины ДТП.

Цель работы --- проанализировать существующие методы распознавания действий водителя по фото.

Для достижения поставленной цели требуется выполнить следующий задачи:
\begin{itemize}[leftmargin=1.6\parindent, label*=---]
	\item провести обзор существующих методов распознавания действий человека за рулем автомобиля по фото;
	\item классифицировать существующие методы распознавания действий водителя по фото;
	\item сформулировать критерии сравнения методов распознавания действий водителя по фото;
	\item провести сравнительный анализ рассмотренных методов.
\end{itemize}

\pagebreak
